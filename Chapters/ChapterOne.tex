\chapter{Introdu\c{c}\~ao}
\label{chapter:introducao}

\textcolor{red}{Sugere-se apresentar um texto que enuncie em que contexto a pesquisa foi, est\'a sendo ou ser\'a realizada (i.e. qual o modelo conceitual do centro de pesquisa, qual as interrela\c{c}\~oes entre os setores ali envolvidos, etc.). Ademais, deve-se ressaltar os \^ambitos (i.e. \'areas de estudos) da pesquisa.}



\section{Defini\c{c}\~ao do problema}
\label{section:definicaoproblema}

\textcolor{red}{Nesta se\c{c}\~ao o problema \'e claramente identificado. Deve-se apresentar os argumentos que levaram \`a identifica\c{c}\~ao do problema que ser\'a trabalho. Alguns autores podem ser citados com o prop\'osito de explicitar a preocupa\c{c}\~ao da comunidade cient\'ifica ou de parte dela, em tentar resolver o problema dentro dos \^ambitos estabelecidos.}



\section{Objetivo}
\label{section:objetivo}

\textcolor{blue}{
    O objetivo principal dessa pesquisa é determinar o algoritmo de odometria visual que apresenta a melhor acurácia e precisão em um ambiente subaquático marinho. Sendo primordial que o robô subaquático possua a capacidade de aprender sobre o ambiente no qual opera, gerando assim sua trajetória ao longo do percurso, a partir de uma sequencia de imagens baseado em informações obtidas através de uma câmera estéreo utilizando o método proposto de odometria visual.
}

\textcolor{red}{https://link.springer.com/article/10.1007/s40903-015-0032-7}

\textcolor{blue}{
	Os objetivos específicos são:
	\begin{itemize}
		\item Determinar os algoritmos de odometria visual mais citados na literatura;
		\item Estabelecer a especificação de requisitos para um ROV que possa utilizar todos os algoritmos selecionados;
		\item Projetar um ROV que atenda aos requisitos específicos;
		\item Implementar os algoritmos selecionados;
		\item Construir o ROV especificado;
		\item Testar o ROV em ambiente real com cada algoritmo implementado;
		\item Validar os resultados de cada algoritmo;
		\item Comparar os resultados e determinar os algoritmos que apresentam maior acurácia em ambiente marinho;
	\end{itemize}
}


\section{Import\^ancia da pesquisa}
\label{section:importanciapesquisa}

\textcolor{red}{O pesquisador/estudante deve apresentar os aspectos mais relevantes da pesquisa ressaltando os impactos (e.g. cient\'ifico, tecnol\'ogico, econ\^omico, social e ambiental) que a pesquisa causar\'a. Deve-se ter cuidado com a ingenuidade no momento em que os argumentos forem apresentados.}



\section{Motiva\c{c}\~ao}
\label{section:motivacao}

\textcolor{red}{Muito associada \`as se\c{c}\~oes \textbf{Defini\c{c}\~ao do problema} e \textbf{Import\^ancia da pesquisa}, esta se\c{c}\~ao, apesar de ser opcional, pode ser relevante principalmente em se tratando de projetos de disserta\c{c}\~oes de mestrado e teses de doutorado. Pode-se apresentar motiva\c{c}\~oes pessoais, baseadas em dados e em interesses do(s) centro(s) de pesquisa onde se est\'a desenvolvendo o projeto.}



\section{Limites e limita\c{c}\~oes}
\label{section:limiteslimitacoes}

\textcolor{red}{Nesta se\c{c}\~ao o pesquisador apresenta os limites e limita\c{c}\~oes definidas antes e e durante o desenvolvimento da pesquisa. Assim, o trabalho fica resguardado de poss\'iveis cr\'iticas. Os limites s\~ao aspectos que estabelecem o escopo da pesquisa e as limita\c{c}\~oes s\~ao os problemas enfrentados durante a pesquisa que levam a tomadas de decis\~ao que podem, inclusive, mudar o direcionamento da pesquisa.}



\section{Quest\~oes e hip\'oteses}
\label{section:questoeshipoteses}

\textcolor{red}{Esta \'e uma se\c{c}\~o importante, pois nela as quest\~oes feitas a partir da defini\c{c}\~ao do problema (comentada na Se\c{c}\~ao \ref{section:definicaoproblema}) s\~ao explicitamente expressadas, conduzindo \`a n\~ao menos expl\'icita, hip\'otese (ou hip\'oteses) que ser\'a investigada, aceita ou rejeitada.}



\section{Aspectos metodol\'ogicos}
\label{section:aspectosmetodologicos}

\textcolor{red}{Aqui devem ser apresentados as assun\c{c}\~oes do pesquisador com rela\c{c}\~ao \`as perspectivas filos\'oficas, aos m\'etodos de pesquisa utilizados e aos modos de an\'alise. Assim, ainda que apresente resultados ou achados pol\^emicos, a pesquisa poder\'a ser validada, pois estar\'a consistente e apresentar\'a uma coer\^encia com as assun\c{c}\~oes do pesquisador.
 uando para a pesquisa \'e muito importante ressaltar os fundamentos metodol\'ogicos, sugere-se escrever ou transformar esta se\c{c}\~ao em um cap\'itulo espec\'ifico.}



\section{Organiza\c{c}\~ao da \thetypework}
\label{section:organizacao}

\textcolor{red}{Este documento apresenta $x$ cap\'itulos\footnote{A quantidade de cap\'itulos desta parte depende da profundidade e necessidade dos \^ambitos da pesquisa (e.g. o cap\'itulo sobre a revis\~ao da literatura pode ser dividos em cap\'itulos: um para cada tema de acordo com o grau de import\^ancia).} e est\'a estruturado da seguinte forma:}

\begin{itemize}

  \item \textbf{Cap\'itulo \ref{chapter:introducao} - Introdu\c{c}\~ao}: Contextualiza o \^ambito, no qual a
  pesquisa proposta est\'a inserida. Apresenta, portanto, a
  defini\c{c}\~ao do problema, objetivos e justificativas da pesquisa e
  como esta \thetypeworkthree est\'a estruturada;

  \item \textbf{Cap\'itulo \ref{chapter:tema1} - Nome do cap\'itulo}: XXX;

  \item \textbf{Cap\'itulo \ref{chapter:tema2} - Nome do cap\'itulo}: XXX;

  \item \textbf{Cap\'itulo \ref{chapter:trabalhoexperimental} - Nome do cap\'itulo}: XXX;

  \item \textbf{Cap\'itulo \ref{chapter:outro} - Nome do cap\'itulo}: XXX;

  \item \textbf{Cap\'itulo \ref{chapter:consideracoesfinais} - Considera\c{c}\~oes Finais}: Apresenta as conclus\~oes, contribui\c{c}\~oes   e algumas sugest\~oes de atividades de pesquisa a serem desenvolvidas no futuro.

\end{itemize}
