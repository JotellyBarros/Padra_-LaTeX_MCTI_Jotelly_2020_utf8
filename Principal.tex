%%%*******************************************************************************
%  Classe Latex SENAI CIMATEC, Version 0.0.0 16/04/2009
%
%  Define as normas e estilo das dissertaes e teses do SENAI CIMATEC
%
%  Author : Hernane Pereira
%%%-------------------------------------------------------------------------------

%%%-------------------------------------------------------------------------------
% Changes log
%%%-------------------------------------------------------------------------------
% Version 1.1.1 - 08/03/2005
% - Added support for amsmath, amsthm, amsfonts and amssymb as default
%
% Version 1.1.0 - 22/07/2004
% - updated university front page format
%
% Version 1.1.0 - 20/06/2003
% - Initial version
%%%-------------------------------------------------------------------------------

%%%-------------------------------------------------------------------------------
%  DOCUMENTATION WITH EXAMPLE
%%%*******************************************************************************

%%%-------------------------------------------------------------------------------
%%% Thesis default options
%%%-------------------------------------------------------------------------------
%\documentclass[subook]{Classes/PPGMCTI}
%\documentclass[sureport]{Classes/PPGMCTI}

%%%-------------------------------------------------------------------------------
%%% Thesis custom options
%%%-------------------------------------------------------------------------------

%%% Fancy page headings
%\documentclass[fancyheadings, subook]{Classes/PPGMCTI}
%\documentclass[fancyheadings, sureport]{Classes/PPGMCTI}

%%% Fancy chapters and sections headings
%\documentclass[fancychapter, subook]{Classes/PPGMCTI}
%\documentclass[fancychapter, sureport]{Classes/PPGMCTI}

%%% Fancy page , chapters and sections headings
%\documentclass[fancyheadings, fancychapter, subook]{Classes/PPGMCTI}
%\documentclass[fancyheadings, fancychapter, sureport]{Classes/PPGMCTI}
\documentclass[fancyheadings, fancychapter, sureport]{Classes/PPGMCTI}

%%%-------------------------------------------------------------------------------
%%% Thesis Commands (ONLY with fancy page headings)
%%%-------------------------------------------------------------------------------

%%%Page header line width
%\footlinewidth{value}

%%%Page footer line width
%\headlinewidth{value}

%%%Page header and footer line width
%\headingslinewidth{value}

%%%Page header and footer lines without text
%\headingslinesonly

%%%The default line width is 0.3pt.
%%%Set the value to 0pt to remove the page header and/or footer line

%%%-------------------------------------------------------------------------------
%%% SUThesis Supported Graphic Formats
%%%-------------------------------------------------------------------------------
% The figures formats supported depend upon the selected output file
% Include your figure without the extention, the SUThesis will automatically
% search the predefined `Figures' directory tree for the right file format.
%
% - The pdfLaTEX (PDF) supports graphics inclusions in PDF, JPG, PNG, and
%   MetaPost (with .mps extention) formats.
%
% - The Latex (DVI) supports graphics inclusions in EPS and PS formats.
%%%-------------------------------------------------------------------------------


%%%-------------------------------------------------------------------------------
%%% rvore de diretrio PPGMCTI
%%%-------------------------------------------------------------------------------
%  Diretrio
%       \Classes        (requerido)
%       \Figures        (requerido) --------------------------------->
%       \Figures\PDF    (optional)
%       \Figures\JPG    (optional) Figures located within these
%       \Figures\PNG    (optional) folders are searched automatically
%       \Figures\MPS    (optional)  by the PPGMCTI class.
%       \Figures\EPS    (optional)
%       \Figures\PS     (optional) <--------------------------------
%       \Tables         (requerido)
%       \Others         (requerido)
%       \Chapters       (requerido)
%       \Appendices     (optional)
%       \References     (requerido)
%%%-------------------------------------------------------------------------------

%%%-------------------------------------------------------------------------------
%%% PDF File Summary
%%%-------------------------------------------------------------------------------
\ifpdf
\hypersetup{backref,
            colorlinks  = true,
            pdftitle    = Titulo da dissertacao ou tese,
            pdfauthor   = {Jotelly Barros Oliveira, jotelly@gmail.com},
            pdfsubject  = Mestre em MCTI,
            pdfcreator  = Subtitulo,
            pdfproducer = PDFLatex,
            pdfkeywords = {Visual Odometry, Egomotion, Robot Vision, SLAM, Underwater}
            }
\fi

%%%-------------------------------------------------------------------------------
%%% Required packages
%%%-------------------------------------------------------------------------------
% - ifthen
% - setspace
% - amsmath
% - amsfonts
% - amssymb
% - amsthm
% - eucal
% - graphics
% - fancyhdr
%%%-------------------------------------------------------------------------------

%%%-------------------------------------------------------------------------------
%%% Optional packages
%%%-------------------------------------------------------------------------------
\usepackage[T1]{fontenc}
\usepackage[utf8]{inputenc}
\usepackage{longtable}
\usepackage{dcolumn}
\usepackage{multirow}
\usepackage{lscape}
\usepackage{graphicx}
\usepackage{rotating}
\usepackage{float,subfigure}
\usepackage[left=3cm,top=3cm,right=2cm,bottom=2cm]{geometry}
\usepackage[alf,abnt-etal-list=5]{abntcite}
\usepackage{ifpdf}
\usepackage{shadow}
\usepackage{wrapfig}
\usepackage[normalem]{ulem}
\usepackage{makeidx} % cria indice remissivo
\usepackage{yfonts}
\usepackage{pdfpages}
\makeindex % cria o indice remissivo
%Tables and Figures Caption
\setlength{\LTcapwidth}{\textwidth}






%\newtheorem{theorem}{Teorema}
%\newtheorem{definition}[theorem]{Defini\c{c}\~ao}


%%%-------------------------------------------------------------------------------
%%% Start thesis root document
%%%-------------------------------------------------------------------------------
\begin{document}
%%----------------------------------------------------------------------------
%% Define the title page
%%----------------------------------------------------------------------------
\university{Centro Universitário SENAI - BAHIA}
\faculty{Programa de Mestrado em Modelagem Computacional e Tecnologia Industrial}
\course{Mestrado em Modelagem Computacional e Tecnologia Industrial}
\typework{Projeto de Dissertação de Mestrado}
\thesistitle{Estudo comparativo de métodos do odometria visual aplicáveis a veículos autônomos subaquáticos}
\hidevolume
\thesisvolume{Volume 1 of 1}
\thesisauthor{Jótelly Barros Oliveira}
\thesisadvisor{Roberto Luiz Souza Monteiro}
\hidecoadvisor
\thesiscoadvisor{Ingrid Winkler}
\universitycoadv{IES}
\thesisdegreetitle{Mestre em Modelagem Computacional e Tecnologia Industrial}
\thesismonthyear{Fevereiro de 2020}

% Defina o nome do ROV
\def\rovname{LowcostROV}

\maketitlepage

%\maketitlepage

%%----------------------------------------------------------------------------
%% Inserir Folha de rosto, Nota de estilo, folha de assinaturas, dedicatoria
%%----------------------------------------------------------------------------
%==========================================================
%        Document: PPGMCTI
%        File....: FolhaRosto.TEX
%        Author..: Hernane Pereira
%        Date....: 16 de abril de 2009
%==========================================================
\begin{folharosto}

\begin{center}
\theauthor
\end{center}
\ \\
\ \\
\ \\
\ \\
\ \\
\begin{spacing}{2}
   \begin{center}
   {\LARGE {\bf \thetitle}}
   \end{center}
\end{spacing}
\ \\
\ \\
\ \\
\begin{flushright}

   \begin{list}{}{
      \setlength{\leftmargin}{4.5cm}
      \setlength{\rightmargin}{0cm}
      \setlength{\labelwidth}{0pt}
      \setlength{\labelsep}{\leftmargin}}

      \item \thetypework apresentada ao \thefaculty, Curso de \thecourse
      do \theuniversity, como requisito parcial para a obten\c{c}\~ao do
      t\'itulo de {\bf \thedegreetitle}.

      \begin{list}{}{
      \setlength{\leftmargin}{0cm}
      \setlength{\rightmargin}{0cm}
      \setlength{\labelwidth}{0pt}
      \setlength{\labelsep}{\leftmargin}}

      \item \'Area de conhecimento: Interdisciplinar

      \item Orientador: \theadvisor
      \newline \hspace*{2.1cm}  {\it \theuniversity}
      \item Coorientador: \thecoadvisor
      \newline \hspace*{2.1cm}  {\it \theuniversitycoadv}


      \end{list}
   \end{list}

\end{flushright}
\ \\
\ \\
\ \\
\ \\
%\begin{spacing}{1.5}
   \begin{center}
   Salvador \par
   \theuniversity \par
   2019
   \end{center}
%\end{spacing}

\end{folharosto}

\include{Others/NotaEstilo}
%\include{Others/FolhaAssinaturas}
\include{Others/dedicatoria}
\include{Others/agradecimentos}


%%----------------------------------------------------------------------------
%% Resumo/abstract, sumrio e siglas
%%----------------------------------------------------------------------------
\begin{romanpagenumbers}
    \begin{thesisresumo}

Este projeto de pesquisa destina-se a análise comparativa de métodos de odometria visual, de modo a determinar quais os que melhor se adequá a um veículo autônomo submersível. de modo a garantir acurácia e eficácia num processo de inclusão e exclusão de pesquisas previamente realizadas disponíveis nas bases de referência bibliográficas (i.e. IEEE, Scopus, Web of science), criou-se um método para a análise e seleção de documentos para revisão sistemática de literatura, para alcançarmos o objetivo mais amplo deste projeto, foi realizada a construção de um protótipo (ROV, veículo submarino operado remotamente) de baixo custo, para comparar os resultados dos algoritmos.

\ \\
\textbf{Palavras-chave:} Visual Odometry, Egomotion, Robot Vision, SLAM, Underwater.
\ \\

\end{thesisresumo}

    \begin{thesisabastract}

This research project is intended for the comparative analysis of visual odometry methods, in order to determine which ones are best suited to an autonomous submersible vehicle. in order to ensure accuracy and efficiency in a process of inclusion and exclusion of previously performed research available in the bibliographic reference bases (ie IEEE, Scopus, Web of science), a method was created for the analysis and selection of documents for systematic review of literature, in order to achieve the broader objective of this project, a low-cost prototype (ROV, underwater operated vehicle) was built to compare the results of the algorithms.

\ \\
\textbf{Keywords:} Visual Odometry, Egomotion, Robot Vision, SLAM, Underwater.
\ \\

\end{thesisabastract}

    % Make list of contents, tables and figures
    \thesiscontents
    %Include other required section
    \begin{thesisabbreviations}
\begin{footnotesize}
\begin{longtable}[l]{p{2cm}l}
    \textbf{Abreviação} & \textbf{Descrição} \\ \\
    DOF 	\dotfill & Graus De Liberdade (Degrees Of Freedom) \\
    DVL  	\dotfill & Registro De Velocidade Doppler (Doppler Velocity Log) \\
    GNSS 	\dotfill & Sistema Global De Navegação Por Satélite (Global Navigation Satellite System) \\
    GPS		\dotfill & Sistema de Posicionamento Global (Global Positioning System) \\
    IMU		\dotfill & Unidades de medidas inerciais (Inertial Measurements Units) \\
    IMU		\dotfill & Unidades de Medição Inercial (Inertial Measurements Units) \\
    INS		\dotfill & Sistemas De Navegação Inercial (Inertial Navigation System) \\
    MEMS 	\dotfill & Micro Sistemas Eletromecânicos (Micro Sistemas Eletromecânicos) \\
    ROV		\dotfill & Veículo Submarino Operado Remotamente (Remotely Operated Underwater Vehicle) \\
    SFM 	\dotfill & Estrutura Do Movimento (Structure From Motion) \\
    SIFT 	\dotfill & Transformação De Recurso Invariante Em Escala (Scale Invariant Feature Transform) \\
    SLAM	\dotfill & Localização E Mapeamento Simultâneos (Simultaneous Localization and Mapping) \\
    VO		\dotfill & Odometria Visual (Visual Odometry) \\    
\end{longtable}
\end{footnotesize}
\end{thesisabbreviations}

    \begin{thesissymbols}
\begin{footnotesize}
\begin{longtable}[l]{p{2cm}l}
    \textbf{Símbolo} & \textbf{Descrição} \\ \\
    $\Omega$          & Unidade de medida da resistência elétrica \\
    $\psi$            & Função de análise wavelet \\
    $\pi$             & Número pi \\
\end{longtable}
\end{footnotesize}
\end{thesissymbols}

    %Switch the page numbering back to the default format
\end{romanpagenumbers}

%%----------------------------------------------------------------------------
%% Include thesis chapters
%%----------------------------------------------------------------------------
\parskip=\baselineskip
\chapter{Introdu\c{c}\~ao}
\label{chapter:introducao}

\textcolor{red}{Sugere-se apresentar um texto que enuncie em que contexto a pesquisa foi, est\'a sendo ou ser\'a realizada (i.e. qual o modelo conceitual do centro de pesquisa, qual as interrela\c{c}\~oes entre os setores ali envolvidos, etc.). Ademais, deve-se ressaltar os \^ambitos (i.e. \'areas de estudos) da pesquisa.}



\section{Defini\c{c}\~ao do problema}
\label{section:definicaoproblema}

\textcolor{red}{Nesta se\c{c}\~ao o problema \'e claramente identificado. Deve-se apresentar os argumentos que levaram \`a identifica\c{c}\~ao do problema que ser\'a trabalho. Alguns autores podem ser citados com o prop\'osito de explicitar a preocupa\c{c}\~ao da comunidade cient\'ifica ou de parte dela, em tentar resolver o problema dentro dos \^ambitos estabelecidos.}



\section{Objetivo}
\label{section:objetivo}

\textcolor{blue}{
    O objetivo principal dessa pesquisa é determinar o algoritmo de odometria visual que apresenta a melhor acurácia e precisão em um ambiente subaquático marinho. Sendo primordial que o robô subaquático possua a capacidade de aprender sobre o ambiente no qual opera, gerando assim sua trajetória ao longo do percurso, a partir de uma sequencia de imagens baseado em informações obtidas através de uma câmera estéreo utilizando o método proposto de odometria visual.
}

\textcolor{red}{https://link.springer.com/article/10.1007/s40903-015-0032-7}

\textcolor{blue}{
	Os objetivos específicos são:
	\begin{itemize}
		\item Determinar os algoritmos de odometria visual mais citados na literatura;
		\item Estabelecer a especificação de requisitos para um ROV que possa utilizar todos os algoritmos selecionados;
		\item Projetar um ROV que atenda aos requisitos específicos;
		\item Implementar os algoritmos selecionados;
		\item Construir o ROV especificado;
		\item Testar o ROV em ambiente real com cada algoritmo implementado;
		\item Validar os resultados de cada algoritmo;
		\item Comparar os resultados e determinar os algoritmos que apresentam maior acurácia em ambiente marinho;
	\end{itemize}
}


\section{Import\^ancia da pesquisa}
\label{section:importanciapesquisa}

\textcolor{red}{O pesquisador/estudante deve apresentar os aspectos mais relevantes da pesquisa ressaltando os impactos (e.g. cient\'ifico, tecnol\'ogico, econ\^omico, social e ambiental) que a pesquisa causar\'a. Deve-se ter cuidado com a ingenuidade no momento em que os argumentos forem apresentados.}



\section{Motiva\c{c}\~ao}
\label{section:motivacao}

\textcolor{red}{Muito associada \`as se\c{c}\~oes \textbf{Defini\c{c}\~ao do problema} e \textbf{Import\^ancia da pesquisa}, esta se\c{c}\~ao, apesar de ser opcional, pode ser relevante principalmente em se tratando de projetos de disserta\c{c}\~oes de mestrado e teses de doutorado. Pode-se apresentar motiva\c{c}\~oes pessoais, baseadas em dados e em interesses do(s) centro(s) de pesquisa onde se est\'a desenvolvendo o projeto.}



\section{Limites e limita\c{c}\~oes}
\label{section:limiteslimitacoes}

\textcolor{red}{Nesta se\c{c}\~ao o pesquisador apresenta os limites e limita\c{c}\~oes definidas antes e e durante o desenvolvimento da pesquisa. Assim, o trabalho fica resguardado de poss\'iveis cr\'iticas. Os limites s\~ao aspectos que estabelecem o escopo da pesquisa e as limita\c{c}\~oes s\~ao os problemas enfrentados durante a pesquisa que levam a tomadas de decis\~ao que podem, inclusive, mudar o direcionamento da pesquisa.}



\section{Quest\~oes e hip\'oteses}
\label{section:questoeshipoteses}

\textcolor{red}{Esta \'e uma se\c{c}\~o importante, pois nela as quest\~oes feitas a partir da defini\c{c}\~ao do problema (comentada na Se\c{c}\~ao \ref{section:definicaoproblema}) s\~ao explicitamente expressadas, conduzindo \`a n\~ao menos expl\'icita, hip\'otese (ou hip\'oteses) que ser\'a investigada, aceita ou rejeitada.}



\section{Aspectos metodol\'ogicos}
\label{section:aspectosmetodologicos}

\textcolor{red}{Aqui devem ser apresentados as assun\c{c}\~oes do pesquisador com rela\c{c}\~ao \`as perspectivas filos\'oficas, aos m\'etodos de pesquisa utilizados e aos modos de an\'alise. Assim, ainda que apresente resultados ou achados pol\^emicos, a pesquisa poder\'a ser validada, pois estar\'a consistente e apresentar\'a uma coer\^encia com as assun\c{c}\~oes do pesquisador.
 uando para a pesquisa \'e muito importante ressaltar os fundamentos metodol\'ogicos, sugere-se escrever ou transformar esta se\c{c}\~ao em um cap\'itulo espec\'ifico.}



\section{Organiza\c{c}\~ao da \thetypework}
\label{section:organizacao}

\textcolor{red}{Este documento apresenta $x$ cap\'itulos\footnote{A quantidade de cap\'itulos desta parte depende da profundidade e necessidade dos \^ambitos da pesquisa (e.g. o cap\'itulo sobre a revis\~ao da literatura pode ser dividos em cap\'itulos: um para cada tema de acordo com o grau de import\^ancia).} e est\'a estruturado da seguinte forma:}

\begin{itemize}

  \item \textbf{Cap\'itulo \ref{chapter:introducao} - Introdu\c{c}\~ao}: Contextualiza o \^ambito, no qual a
  pesquisa proposta est\'a inserida. Apresenta, portanto, a
  defini\c{c}\~ao do problema, objetivos e justificativas da pesquisa e
  como esta \thetypeworkthree est\'a estruturada;

  \item \textbf{Cap\'itulo \ref{chapter:tema1} - Nome do cap\'itulo}: XXX;

  \item \textbf{Cap\'itulo \ref{chapter:tema2} - Nome do cap\'itulo}: XXX;

  \item \textbf{Cap\'itulo \ref{chapter:trabalhoexperimental} - Nome do cap\'itulo}: XXX;

  \item \textbf{Cap\'itulo \ref{chapter:outro} - Nome do cap\'itulo}: XXX;

  \item \textbf{Cap\'itulo \ref{chapter:consideracoesfinais} - Considera\c{c}\~oes Finais}: Apresenta as conclus\~oes, contribui\c{c}\~oes   e algumas sugest\~oes de atividades de pesquisa a serem desenvolvidas no futuro.

\end{itemize}

\chapter{Revis\~ao da literatura especializada - Fundamenta\c{c}\~ao te\'orica}
\label{chapter:odometriavisual}
%\setlength{\headheight}{52pt} % ...at least 51.60004pt
%----------------------------------------------------------------------

\textcolor{blue}{Esta seção deve conter a Revisão Bibliográfica da sua proposta de dissertação de acordo com as instruções normativas descritas.}

\section{Odometria Visual}
\label{sec:odometriavisual}

A odometria visual que em inglês significa visual odometry (VO), também conhecida como egomotion, é abordado na área da visão computacional como sendo o processo para  estimar a posição atual (pose) de um agente (por exemplo, veículo, humano e robô) usando apenas a entrada de uma ou várias câmeras anexadas a ele. Domínios de aplicação incluem robótica, computação vestível, realidade aumentada e automotiva \cite{fraundorfer2011visual}. O VO é um caso particular de uma técnica conhecida como Estrutura do Movimento (SFM)  que aborda o problema da reconstrução 3D da estrutura do ambiente e das poses da câmera a partir de conjuntos de imagens sequencialmente ordenados ou não-ordenados \cite{yousif2015overview}.

De acordo com \cite{fraundorfer2011visual}, o termo Visual Odometry (VO) foi cunhado em 2004 por Nister em seu documento de referência \cite{nister2004visual}. O termo foi escolhido por sua semelhança com a odometria das rodas, que estima incrementalmente o movimento de um veículo, integrando o número de voltas de suas rodas ao longo do tempo. Da mesma forma, o VO opera estimando incrementalmente a pose do veículo através do exame das mudanças que o movimento induz nas imagens de seus sensores a bordo. A vantagem do VO em relação à odometria da roda é que o VO não é afetado pelo deslizamento da roda em terrenos irregulares ou outras condições adversas.

Foi demonstrado que, em comparação com a odometria da roda, o VO fornece estimativas de trajetória mais precisas, com erros de posição relativa variando de 0,1 a 2\% \cite{wirth2013visual}, \cite{fraundorfer2011visual} - \cite{nawaf2017towards}. Essa capacidade faz do VO um complemento interessante à odometria das rodas e, adicionalmente, a outros sistemas de navegação, como as Unidades de Medição Inercial (IMUs).

\subsection{Odometria Visual Subaquática}
\label{sec:odometriavisualsubaquatica}

A técnica de VO está se tornando popular nos AUVs para navegação, manutenção de estações e fornecimento de informações de feedback para manipulação.
A saída da odometria visual é frequentemente combinada com as IMUs para fornecer uma alternativa mais barata aos DVLs e sistemas de navegação inercial (INS) \cite{bellavia2017selective}, \cite{wirth2013visual} e \cite{yousif2015overview}. No entanto, as limitações da visão subaquática são amplamente conhecidas e seu desempenho depende de muitos fatores, como visibilidade, iluminação e distorção resultantes de diferentes índices de refração. De acordo com \cite{fraundorfer2011visual}, para que o VO funcione efetivamente, deve haver iluminação suficiente no ambiente e uma cena estática com textura suficiente para permitir a extração de movimentos aparentes. Além disso, os quadros consecutivos devem ser capturados, garantindo que eles tenham sobreposição de cena suficiente.
%% ------------------------------------------------------------------------- %%

\section{Métodos de Odometria Visual}
\label{sec:metodosdeodometriavisual}

Nos últimos anos, muitos métodos de VO foram propostos, que podem ser divididos em métodos de câmera monocular e estéreo \cite{yousif2015overview}. Esses métodos são posteriormente divididos em correspondência de recursos (recursos correspondentes em vários quadros), rastreamento de recursos (recursos correspondentes em quadros adjacentes) e técnicas de fluxo óptico (com base na intensidade de todos os pixels ou regiões específicas nas imagens sequenciais). De acordo com \cite{wirth2013visual}, o pipeline básico do algoritmo de odometria visual consiste nas seguintes etapas (independentemente do tipo de câmera): primeiro, os pontos-chave são identificados em cada quadro da câmera e os descritores de recursos para esses pontos são extraídos. Em seguida, a profundidade de cada ponto de referência é estimada usando estéreo, estrutura de movimento ou uma câmera de profundidade separada. Posteriormente, os recursos são correspondidos nos prazos e a transformação de corpo rígido que melhor alinha os recursos entre os quadros é estimada. O resultado desse processo é uma estimativa do movimento da câmera entre os quadros e, portanto, é necessário integrar esses dados ao longo do tempo para obter a posição e orientação absolutas do veículo. Normalmente, esse resultado final é refinado com uma otimização offline (ou seja, ajuste de pacote), cujo tempo de computação cresce com o número de imagens \cite{fraundorfer2011visual}, \cite{nawaf2017towards}. A Figura~\ref{fig:Figures/dgblocoVO} apresenta uma sequência típica para algoritmos VO de acordo com \cite{yousif2015overview}.
\ \\
\begin{figure}[!htb]
	\centering	
	\includegraphics[width=0.5\textwidth]{Figures/DiagramaDeBlocosVO.png}
	\caption{Um diagrama de blocos mostrando os principais componentes de um componente de odometria visual \cite{yousif2015overview}.}
	\label{fig:Figures/dgblocoVO}
\end{figure}
\ \\
%% ------------------------------------------------------------------------- %%

\subsection{Visual Monocular}
\label{sec:visualmonocular}

De acordo com \cite{yousif2015overview}, no VO monocular, os pontos de recurso precisam ser observados em pelo menos três quadros diferentes (observe os recursos no primeiro quadro, re-observe e triangule em pontos 3D no segundo quadro e calcule a transformação no terceiro quadro, Armação). Uma questão importante no VO monocular é o problema de ambiguidade da escala. Ao contrário dos sistemas de visão estéreo em que a transformação (rotação e translação) entre os dois primeiros quadros da câmera pode ser obtida, a transformação entre os dois primeiros quadros consecutivos na visão monocular não é totalmente conhecida (a escala é desconhecida) e geralmente é definida como um valor predefinido . Em \cite{wirth2013visual}, esses problemas também são considerados e descritos. Segundo o autor, sistemas que usam apenas uma câmera precisam de movimento de translação para estimativa de movimento em 3D e todas as medições precisam ser dimensionadas por um fator desconhecido para estar em uma escala métrica.

O uso de uma câmera estéreo calibrada supera os dois problemas mencionados, pois as coordenadas 3D dos pontos correspondentes em um único par de imagens esquerda/direita podem ser calculadas por triangulação.

A Figura~\ref{fig:Figures/MonocularVOSystem} apresenta as poses relativas entre as câmeras que visualizam o mesmo ponto 3D são calculadas combinando os pontos correspondentes na imagem 2D. Se a localização 3D dos pontos for conhecida, pode ser usado um método 3D para 3D ou 3D para 2D. As poses globais são calculadas concatenando as transformações relativas em relação a um quadro de referência (pode ser definido como o quadro inicial) \cite{yousif2015overview}
\ \\
\begin{figure}[!htb]
	\centering	
	\includegraphics[width=0.6\textwidth]{Figures/MonocularVOSystem.png}
	\caption{Um exemplo de sistema VO monocular.}
	\label{fig:Figures/MonocularVOSystem}
\end{figure}
\ \\
As abordagens de VO foram apresentadas também para o domínio subaquático. Usando pipelines que incluem rastreamento de recursos e estimativa de movimento, esses sistemas são capazes de calcular transformações planares \cite{huang2017visual} ou seis graus de liberdade (DoF) \cite{wirth2013visual}, \cite{corke2007experiments} incrementais da pose da câmera. Medidas inerciais também podem ser incluídas para melhorar o desempenho \cite{creuze2017monocular}.

De acordo com \cite{bellavia2017selective}, os ambientes subaquáticos são tipicamente caracterizados por imagens não estruturadas, ruidosas e altamente texturizadas, com padrões repetitivos e más condições de iluminação local (efeitos de vinheta e outros artefatos). Consequentemente, os pontos-chave da imagem são difíceis de rastrear e corresponder corretamente debaixo d'água, mesmo se forem empregados detectores/descritores de recursos estáveis e robustos, como a transformação de recurso invariante em escala (SIFT) \cite{lowe2004distinctive}, os recursos robustos acelerados (SURF) \cite{bay2006surf} ou descritores de recursos com base nos momentos de Zernike \cite{eustice2008visually}, \cite{kim2009pose}.

A maioria das soluções subaquáticas propostas prefere configurações estéreo em vez de monoculares para melhorar a robustez e a precisão da saída, evitando problemas como os citados antes \cite{bellavia2017selective}, \cite{kim2013real}.
%% ------------------------------------------------------------------------- %%

\subsection{Visual Estéreo}
\label{sec:visualestereo}
\ \\
Usando o mesmo conceito do sistema visual humano, o sistema de visão estereoscópica ou estereoscópica emprega um conjunto de câmeras binoculares, como mostrado na Figura~\ref{fig:Figures/ImagensSimutaneas}. Segundo os autores em \cite{stivanello2008correspondencia}, um sistema estéreo pode extrair informações da cena observada recuperando dados de profundidade de um ponto no espaço a partir da distância relativa entre dois pontos. Essa diferença nas coordenadas dos pontos correspondentes às projeções é chamada disparidade.
\ \\
\begin{figure}[!htb]
	\centering	
	\includegraphics[width=0.7\textwidth]{Figures/ImagensSimutaneas.png}
	\caption{Imagem de duas lentes simultâneas \cite{stivanello2008correspondencia}.}
	\label{fig:Figures/ImagensSimutaneas}
\end{figure}
\ \\
A geometria necessária para fornecer a terceira dimensão é baseada no conceito de geometria epipolar, onde duas câmeras são apontadas para a mesma cena em duas posições distintas, resultando em várias conexões geométricas entre elas e, portanto, fornecendo os pontos 3D, epipolar. A geometria é apresentada com mais detalhes em \cite{bappy2012study}.

\begin{figure}[!htb]
	\centering	
	\includegraphics[width=0.5\textwidth]{Figures/EpipolarGeometry.png}
	\caption{Conceito de geometria epipolar \cite{bappy2012study}.}
	\label{fig:Figures/EpipolarGeometry}
\end{figure}
\ \\
A odometria visual (VO) é o processo de estimar o movimento de um veículo em movimento usando a entrada de vídeo de suas câmeras a bordo.

No VO estéreo, o movimento é estimado pela observação de recursos em dois quadros sucessivos (nas imagens direita e esquerda). Na Figura~\ref{fig:Figures/Clound3DPoints}, são mostrados quadros rastreados para formar uma nuvem de pontos 3D esparsos do ambiente e podem ser usados como base para a localização.
\ \\
\begin{figure}[!htb]
	\centering	
	\includegraphics[width=0.8\textwidth]{Figures/Clound3DPoints.png}
	\caption{Os quadros rastreados formam uma nuvem de pontos 3D \cite{stivanello2008correspondencia}.}
	\label{fig:Figures/Clound3DPoints}
\end{figure}
\ \\
De acordo com os autores em \cite{stivanello2008correspondencia}, uma das principais vantagens da visão estéreo é que as imagens armazenam uma quantidade enorme de informações significativas, mesmo sem a necessidade de mover a câmera, além disso, fornecem alta precisão da localização, provando ser barata. solução em comparação com outros sensores de proximidade. Em \cite{zhou2017robust}, devido à certa linha de base entre as câmeras esquerda e direita, o problema de escala de ambiguidade é estabelecido e provado que não existe no VO estéreo. Além disso, o VO estéreo pode estimar o 6-Degree of Freedom (DOF) e o sensor do movimento do ego, o que não diz respeito ao tipo de ambiente em que o sistema trabalha. Atualmente, existem dois tipos de métodos de VO estéreo, 3D-3D e 3D-2D.

De acordo com \cite{yousif2015overview}, essa estimativa de movimento 3D-3D é realizada triangulando pontos de recurso 3D observados em uma sequência de imagens. A transformação entre os quadros da câmera é então estimada minimizando a distância euclidiana 3D entre os pontos 3D correspondentes.

Os autores também discutiram sobre o método de estimativa de movimento 3D-2D semelhante à abordagem anterior, mas aqui o erro de re-projeção 2D é minimizado para encontrar a transformação necessária. Novamente, o número mínimo de pontos requeridos varia com base no número de restrições no sistema.
Embora esses métodos de VO estéreo tenham se mostrado promissores, existem alguns problemas em relação à captura de imagens, gerando desvio de trajetória e valores discrepantes, como condições de iluminação, textura ao redor, presença de água, neve, desfoque de movimento, presença de sombras, semelhança visual, configuração degenerada e oclusões.

Com o objetivo de avaliar a melhor abordagem para estimativa de localização, a comparação de desempenho entre as abordagens de VO estéreo e VO monocular foi realizada em \cite{chen2015performance}. Figura~\ref{fig:Figures/DemoPerMonocular}, a trajetória no movimento de curvatura pode ser facilmente distinguida com o caminho de referência no modo monocular. O valor da distância percorrida é de 6,3\%, o principal motivo é que o resultado da correspondência monocular não é robusto o suficiente para fornecer uma boa estimativa de movimento quando o veículo está sob a dinâmica de viragem. Além disso, há um problema de ambiguidade de escala no resultado.

Portanto, o sistema VO estéreo tem desempenho mais estável que o sistema monocular, e a superioridade pode ser verificada através do experimento de tornar dinâmico em \cite{chen2015performance}.

Figura~\ref{fig:Figures/AnaliseOdometria}, é mostrada a comparação de desempenho entre o sistema de integração INS (Sistema de Navegação Inercial)/GNSS (Sistema Global de Navegação por Satélite), que é um dos mais usados na área de navegação, e a abordagem VO (Odometria Visual), respectivamente para sistemas monocular e estéreo. A taxa de amostragem mais alta tem melhor desempenho tanto na trajetória pura quanto na TD (Distância percorrida), tanto E (leste) (m) quanto N (norte) (m), quando o veículo passa por algumas áreas onde a qualidade do sinal GPS recebido é instável, o INS/GPS MEMS (Micro Sistemas Eletromecânicos) com o esquema fracamente acoplado gerará facilmente piores soluções de navegação \cite{chen2015performance}.
\ \\
\begin{figure}[!htb]
	\centering	
	\includegraphics[width=0.6\textwidth]{Figures/AnaliseOdometria.png}
	\caption{A análise numérica de dois sistemas de odometria visual em que a comparação é curva é mostrada na Figura~\ref{fig:Figures/DemoPerMonocular} e na Figura~\ref{fig:Figures/DemoPerEstereo}
	\cite{chen2015performance}.}
	\label{fig:Figures/AnaliseOdometria}
\end{figure}
\ \\
\begin{figure}[!htb]
	\centering	
	\includegraphics[width=0.6\textwidth]{Figures/DemoPerMonocular.png}
	\caption{Demonstração de VO monocular de desempenhos com mais distorção
	\cite{chen2015performance}.}
	\label{fig:Figures/DemoPerMonocular}
\end{figure}
\ \\
\begin{figure}[!htb]
	\centering	
	\includegraphics[width=0.6\textwidth]{Figures/DemoPerEstereo.png}
	\caption{VO estéreo, demonstrativa de desempenho mais estável \cite{chen2015performance}.}
	\label{fig:Figures/DemoPerEstereo}
\end{figure}
\ \\
%% ------------------------------------------------------------------------- %%

\section{Visual SLAM}
\label{sec:visualslam}

De acordo com \cite{yousif2015overview}, \cite{fraundorfer2011visual} SLAM é uma maneira de um robô se localizar em um ambiente desconhecido, enquanto constrói incrementalmente um mapa de seus arredores. O SLAM foi extensivamente estudado nas últimas décadas, resultando em muitas soluções diferentes usando sensores diferentes, incluindo sensores de sonar, sensores de infravermelho e scanners LASER. Recentemente, houve um interesse crescente no SLAM baseado em visual, também conhecido como V-SLAM, devido às informações visuais ricas disponíveis nos sensores de vídeo passivos de baixo custo em comparação aos scanners LASER.

De acordo com \cite{fraundorfer2011visual}, \cite{nister2004visual}, o objetivo do V-SLAM é obter uma estimativa global e consistente do caminho do robô. Isso implica manter um controle de um mapa do ambiente (mesmo no caso em que o mapa não é necessário per se) porque é necessário perceber quando o robô retorna a uma área visitada anteriormente, esse processo é conhecido como detecção de fechamento de loop. Quando um fechamento de loop é detectado, essas informações são usadas para reduzir a deriva no mapa e no caminho da câmera. De acordo com \cite{wirth2013visual}, os algoritmos para odometria visual - em oposição aos algoritmos SLAM completos - concentram-se em estimativas rápidas de movimento quadro a quadro, sem manter um grande histórico de fechamento de loop.

A ênfase está em medições precisas em altas frequências. Entender quando ocorre um fechamento de loop e integrar eficientemente essa nova restrição no mapa atual são dois dos principais problemas do V-SLAM. Uma das principais desvantagens da implementação das técnicas do V-SLAM depende do tempo de computação e utiliza recursos que, por sua vez, crescem significativamente para grandes ambientes \cite{nawaf2017towards}. Em \cite{strasdat2012visual} é apresentada uma pesquisa extensa considerando outros aspectos relacionados ao V-SLAM e aplicações similares.
%% ------------------------------------------------------------------------- %%


%\chapter{Revis\~ao da literatura especializada - Fundamenta\c{c}\~ao te\'orica do Tema 2}
%\label{chapter:tema2}

\chapter{O Projeto \rovname}
\label{chapter:contrucaorov}

Esta seção pode conter a construção do ROV.

\section{Desenvolvimento do Projeto}
\label{sec:desenvolvimentodoprojeto}

%% ------------------------------------------------------------------------- %%
\subsection{Softwares}
\label{sec:softwares}

texto texto texto texto texto texto texto texto texto texto
texto texto texto texto texto texto texto texto texto texto
texto texto texto texto texto texto texto texto texto texto

\subsubsection{Ubuntu Linux 16.04}
\label{sec:ubuntulinux}

De acordo com \cite{whatisUbuntu}, O Ubuntu é um sistema operacional Linux completo, disponível gratuitamente (Open-Source) com suporte comunitário e profissional. O Projeto Ubuntu é patrocinado pela \cite{canonical}. A Canonical não cobrará taxas de licença pelo Ubuntu, agora ou em qualquer estágio no futuro. O modelo de negócios da Canonical é fornecer suporte técnico e serviços profissionais relacionados ao Ubuntu.

\subsubsection{ROS e Gazebo (Ambiente de Simulação)}
\label{sec:rosegazebo}

De acordo com \cite{ROSIntroduction}, O ROS é um sistema meta-operacional de código aberto. Fornece os serviços como abstração de hardware, controle de dispositivo de baixo nível, comumente usado funcionalidade, passagem de mensagens entre processos e pacote gestão.

De acordo com \cite{gazebo}, Gazebo É o simulador de física do mundo real, oferecendo a capacidade de simular com precisão e eficiência populações de robôs em ambientes internos e externos complexos sendo gratuito com uma comunidade ativa.

\subsubsection{Gephi (Pacote de software de análise e visualização de rede)}
\label{sec:gephi}

texto texto texto texto texto texto texto texto texto texto
texto texto texto texto texto texto texto texto texto texto
texto texto texto texto texto texto texto texto texto texto

\subsubsection{Draw.io (Aplicação de diagramação gráfica online)}
\label{sec:drawio}

texto texto texto texto texto texto texto texto texto texto
texto texto texto texto texto texto texto texto texto texto
texto texto texto texto texto texto texto texto texto texto

\subsubsection{Solidworks (Software CAD 3D (Design Assistido por Computador))}
\label{sec:solidwoks}

texto texto texto texto texto texto texto texto texto texto
texto texto texto texto texto texto texto texto texto texto
texto texto texto texto texto texto texto texto texto texto

\subsubsection{Cura (Software fatiador e servidor para impressora 3D)}
\label{sec:cura}

texto texto texto texto texto texto texto texto texto texto
texto texto texto texto texto texto texto texto texto texto
texto texto texto texto texto texto texto texto texto texto
%% ------------------------------------------------------------------------- %%

\subsection{Desenho Técnico}
\label{sec:desenhotecnico}

Apresentação da vista do desenho técnico do \rovname \space desenvolvido e renderizado no SolidWorks, como é mostrado nas figuras Figura~\ref{fig:Projvistatotal}, Figura~\ref{fig:Projvistafrnotal}, Figura~\ref{fig:Projvistainferior} e Figura~\ref{fig:Projvistasuperior}, vista geral, vista frontal, vista inferior e superior respectivamente.

\begin{figure}[!htb]
	\centering	
	\includegraphics[width=0.5\textwidth]{Figures/Rov/Proj_vista_total.jpeg}
	\caption{Vista Total, Fonte: Elaborada pelo autor\the\year}
	\label{fig:Projvistatotal}
\end{figure}

\begin{figure}[!htb]
	\centering	
	\includegraphics[width=0.5\textwidth]{Figures/Rov/Proj_vista_frontal.jpeg}
	\caption{Vista Frontal, Fonte: Elaborada pelo autor\the\year}
	\label{fig:Projvistafrnotal}
\end{figure}

\begin{figure}[!htb]
	\centering	
	\includegraphics[angle=90,width=0.5\textwidth]{Figures/Rov/Proj_vista_inferior.jpeg}
	\caption{Vista Inferior, Fonte: Elaborada pelo autor\the\year}
	\label{fig:Projvistainferior}
\end{figure}

\begin{figure}[!htb]
	\centering	
	\includegraphics[angle=90,width=0.5\textwidth]{Figures/Rov/Proj_vista_superior.jpeg}
	\caption{Vista Superior, Fonte: Elaborada pelo autor\the\year}
	\label{fig:Projvistasuperior}
\end{figure}

%% ------------------------------------------------------------------------- %%

\subsection{Estrutura (Chassi)}
\label{sec:estruturachassi}

texto texto texto texto texto texto texto
texto texto texto texto texto texto texto
texto texto texto texto texto texto texto

Materiais utilizados para a construção.

\begin{itemize}
    \item 1 - Tubo PVC 150mm;
    \item 4 - Bomba De Porão Seaflo 1100gph - 4164lph 12v;
    \item 2 - Tampas PVC 150mm com borracha;
    \item 4 - Suportes impressos em PLA 3d;
    \item 4 - Hélices de plástico;
    \item 1 - Arduíno Mega 2560;
    \item 2 - Ponte H Monster Motor Shield VNH2SP30 30a 2 Motor;
    \item 1 - Cúpula Acrílica Hemisférica;
    \item 40 - Parafusos de aço inox com porca;
    \item 15 - Metros de fios awg22;
\end{itemize}

\begin{figure}[!htb]
	\centering	
	\includegraphics[width=0.5\textwidth]{Figures/Rov/Estrutura_ROV.jpeg}
	\caption{Vista Superior, Fonte: Elaborada pelo autor\the\year}
	\label{fig:estruturarov}
\end{figure}
%% ------------------------------------------------------------------------- %%

\subsection{Motores de Tração}
\label{sec:motoresdetracao}

De acordo com \cite{bombadeporao}, As bombas de esgoto não-automáticas 1100GPH da SEAFLO oferecem evacuação de água ativada por um painel ou interruptor de bóia alem de Incluir proteção de bloqueio anti-ar e vedações herméticas exclusivas, fiação de bloco de grau marítimo. Disponível para tensão 12 e 24 tensão. O modelo 1100GPH 12V, atende as especificações do projeto devido ao seu alto torque e selagem hermética, possibilitando uma submersão de até 20 pés (6,096) metros.

Principais Características:

\begin{itemize}
    \item 2 anos de garantia
    \item Atende ou excede os padrões RoHS, SGS e ISO
    \item Operação tradicional por interruptor ou interruptor de bóia
    \item Totalmente submersível
    \item Base do filtro de encaixe para fácil manutenção
    \item Bomba e interruptor são produtos independentes
    \item Motores eficientes e de longa vida útil selados
    \item Funciona a seco, capaz de cargas de trabalho normais
    \item Operação silenciosa
    \item limites de temperatura de $110\,^{\circ}\mathrm{F}$ ($43\,^{\circ}\mathrm{C}$)
    \item eixo de aço inoxidável
    \item Proteção anti-Airlock
    \item Selos herméticos exclusivos
    \item Fiação bloqueada de grau marítimo
\end{itemize}

\begin{figure}[!htb]
	\centering
	\includegraphics[width=0.8\textwidth]{Figures/Rov/Bomba_porao-Seaflo.jpg}
	\caption{Vista Superior, \cite{bombadeporao}}
	\label{fig:bombaporao}
\end{figure}
%% ------------------------------------------------------------------------- %%

\subsection{Cabos e Conexões}
\label{sec:caboseconexoes}

%% ------------------------------------------------------------------------- %%

\subsection{Sensores}
\label{sec:sensores}
%% ------------------------------------------------------------------------- %%

\subsection{Microcontroladores}
\label{sec:microcontroladores}

O sistema embarcado utilizado para controlar as entradas e saídas de comandos lógicos do \rovname.

\subsubsection{Arduino}
\label{sec:arduino}
(+ texto)

\subsubsection{Raspberry Pi}
\label{sec:raspberrypi}
(+ texto)
%% ------------------------------------------------------------------------- %%

\subsection{Ponte-H}
\label{sec:ponteh}

De acordo com \cite{monstermotorshield}, O VNH2SP30 é um driver de motor de ponte que incorpora um driver monolítico duplo alto e dois interruptores laterais baixos, os pinos do sensor de corrente (CS) produzirão aproximadamente 0,13 volts por amp de corrente de saída. Sendo uma ótima escolha já que o mesmo é capaz de controlar até dois motores com corrente de pico de 30A, controlando o sentido de rotação e velocidade do motor.

Principais Características:

\begin{itemize}
	\item Tensão máxima: 30V;
    \item Máxima corrente: 30 A;
    \item Corrente nominal de operação: 14 A;
    \item Medição de corrente disponível para o Arduino nos pinos analógicos;
    \item MOSFET sobre resistência: 19 m$\Omega$;
    \item Frequência PWM máxima: 20 kHz;
    \item Proteção térmica;
    \item Proteção contra sobretensão e subtensão;
\end{itemize}

\begin{figure}[!htb]
	\centering	
	\includegraphics[width=0.5\textwidth]{Figures/Rov/Monster_motor_shield.jpg}
	\caption{Vista Superior Monster Motor Shield \cite{monstermotorshield}}
	\label{fig:monstermotorshield}
\end{figure}
%% ------------------------------------------------------------------------- %%

\subsection{Fonte de Alimentação}
\label{sec:fontedealimentacao}
%% ------------------------------------------------------------------------- %%

\subsection{Painel de Controle}
\label{sec:paineldecontrole}
%% ------------------------------------------------------------------------- %%

\chapter{Trabalho experimental e desenvolvimento da pesquisa}
\label{chapter:trabalhoexperimental}



\section{Modelo proposto}
\label{section:titulo41}


\section{An\'alise experimental/Simula\c{c}\~oes e cen\'arios}
\label{section:titulo42}


\section{Resultados}
\label{section:titulo43}


\section{Discuss\~ao}
\label{section:titulo4N}

\chapter{Título do capítulo}\label{chapter:outro}

\textcolor{red}{As organizações atuam em um ...}
%\include{Chapters/ChapterSix}
%\include{Chapters/ChapterSeven}
%\include{Chapters/ChapterEight}
\chapter{Considera\c{c}\~oes finais}
\label{chapter:consideracoesfinais}

\textcolor{red}{Chegou a hora de apresentar o apanhado geral sobre o trabalho de
pesquisa feito, no qual s\~ao sintetizadas uma s\'erie de
reflex\~oes sobre a metodologia usada, sobre os achados e
resultados obtidos, sobre a confirma\c{c}\~ao ou recha\c{c}o da
hip\'otese estabelecida e sobre outros aspectos da pesquisa que
s\~ao importantes para validar o trabalho. Recomenda-se n\~ao
citar outros autores, pois a conclus\~ao \'e do pesquisador.
Por\'em, caso necess\'ario, conv\'em cit\'a-lo(s) nesta parte e
n\~ao na se\c{c}\~ao seguinte chamada \textbf{Conclus\~oes}.}


\section{Conclus\~oes}
\label{section:conclusoes}

\textcolor{red}{Brevemente comentada no texto acima, nesta se\c{c}\~ao o
pesquisador (i.e. autor principal do trabalho cient\'ifico) deve
apresentar sua opini\~ao com respeito \`a pesquisa e suas
implica\c{c}\~oes. Descrever os impactos (i.e.
tecnol\'ogicos,sociais, econ\^omicos, culturais, ambientais,
políticos, etc.) que a pesquisa causa. N\~ao se recomenda citar
outros autores.}


\section{Contribui\c{c}\~oes}
\label{section:contribuicoes}

\textcolor{red}{Apresentar as contribui\c{c}\~oes te\'oricas e pr\'aticas que a
pesquisa e o(s) seu(s) resultado(s) trazem.}



\section{Atividades Futuras de Pesquisa}
\label{section:atividadesfuturaspesquisa}

\textcolor{red}{Esta \'e uma se\c{c}\~ao breve, uma vez que o pesquisador
apresenta as id\'eias que ser\~ao desenvolvidas e/ou as atividades
que j\'a est\~ao em andamento, de maneira que a pesquisa realizada
seja complementada ou ampliada. Com esta se\c{c}\~ao, o
pesquisador indica implicitamente que o trabalho n\~ao acabou e
que seu interesse continua ativo.}

% include more chapters ...

%%----------------------------------------------------------------------------
%% Include thesis appendices
%%----------------------------------------------------------------------------

\begin{thesisappendices}
    % Thesis Appendix -------------------------------------------------------

\chapter{Documentos}
\label{Appendice:Documentos}

\textcolor{red}{XXX}

\section{Questionário}

\textcolor{red}{XXX}

\newpage
\section{Diagramas detalhados do modelo}

\textcolor{red}{XXX}


%        \include{Appendices/Fitting}
%        \include{Appendices/Design}
    % include more appendices ...
\end{thesisappendices}

%%----------------------------------------------------------------------------
%% Configurar as referencias bibliograficas
%%----------------------------------------------------------------------------


\addcontentsline{toc}{chapter}{Referencias}
\bibliographystyle{abnt-alf}


\bibliography{References/referencias}

%%----------------------------------------------------------------------------
%% Finishing
%%----------------------------------------------------------------------------

\include{Others/ultimafolha}
\end{document}
%%%-------------------------------------------------------------------------------
%%% Here we finished with your thesis formating. Good luck with the contents
%%%-------------------------------------------------------------------------------
